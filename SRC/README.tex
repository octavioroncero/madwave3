
MAIN Programs: (should contain MPI initialization)
   

MODULES:

1. mod_GridYpara_01y2.f :  First module which initialize paralelization and determines radial and angular grids
                          reads na
         
         contain subroutines:
                                  1.1. input_grid:  Reads /inputgridbase/ namelist in ``input.dat'' file
                                                   with grid and basis data in a 01-2 Jacobi coordinates in body-fixed
                                                   and determines grids and no. of points

                                  1.2. paralelizacion: based on the no. of Omega proyections
                                                   and angular grids, factorize the calculation
                                                   among different processors (nproc) used and initialized
                                                   in the main program
                       
2. mod_pot_01y2.f: Determines the potential and write (pot1) or read (pot2) it

          contains subroutines:
                                  2.1. pot0:  Reads /inputpotmass/ namelist in ``input.dat'' file
                                             to determine the ``masses'' of the 01 + 2 system and ``vcutmaxeV''
                                             and initialize the potential calculation, calling to
                                                               ``setxbcpotele''
                                             to determine features of the electronic states used

                                  2.2. pot1: calculates the potential in the grid, and writes the values
                                             below ``vcutmaxeV'', to reduce the grid, in files pot/pot.IANG.dat
                                             where IANG is an index determininig the angular value in the grid used
                                             Uses
                                                              ``potelebond'' user provide routine to generate potential
                                                                             only used to generate potential
                                                                             so that in normal wave-packet calculations
                                                                             a general version can be used in precompiled 
                                                                             version of the whole program
                                                              ``DIAGON'' in liboctdyn.f library of general purpose provided here
            
3. mod_baseYfunciones.f: Determines basis set functions quantum numbers, angular functions 
                         and radial phi_vj functions of the 01 fragment

          contains subroutines:
                                 
                                  3.1. basis: determines rotational-electronic basis set of the calculations
                                               with the data previously read in ``input_grid'' routine
                                               and distributes them among processors
             
                                  3.2. angular_functions: determines angular functions d^j_mm(gamma_i)
                                               in the angular grid of the calculation.
                                               Uses
                                                     ``dwigner''  in liboctdyn library provided here

                                  3.3. write_radial_functions01: determines radial phi^e_vj(r=R1_i) 
                                               for each electronic state ``d'' considered to be diabatic
                                               at long 01 --- 2 distances. 
                                               Used to project the wavepacket along propagation and determine final
                                               state distribution. Also, one is used to determine initial
                                               wavepacket in collisions from a particular v_{ref} j_{ref}
                                               Uses
                                                       ``tqli'' adapted from Numerical Recipies provided here
                                                       ``schr'' provided in libocdyn
                                                       ``splset,splinq'' provided in liboctdyn
                                               writes the energies and functions in ``cont.bcwf''

                                  3.4. read_radial_functions01:
